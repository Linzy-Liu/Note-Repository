\documentclass{ctexart}
\usepackage{amsmath, amsthm, amssymb, mathrsfs,anyfontsize}

\begin{document}

\section{模拟退火(Stimulated Annealing)}

\subsection{想法}

事实上,\textbf{模拟退火算法}可以视为一种改良后的``贪婪''类算法。它的重大进步就在于通过Metropolis采样让算法能脱离局部最优解,
得到全局最优解。这个算法抽象自金属冶炼的退火过程。被冶炼的物体在由高温缓慢降至低温时,物体自身会自发地使自身的内能降至最低点。
其原理在于高温时,物体内的原子会进行较强的分子热运动,易于跳出自身原来所在的平衡位置。而随着温度的缓慢降低,分子的跳跃会越来越困难。
从而最终达到``全局最低内能''的状态。

\subsection{流程}

\noindent 需给条件:\\
\begin{tabular}[t]{l|l}
    \hline
    初始解 &  $x_0$\\
    迭代数 & $L$ \\
    扰动函数 & $C$\\
    代价函数(内能函数) & $E$\\
    降温函数 & $\text{TransT}$\\
    \hline
\end{tabular}


\begin{enumerate}
    \item 给定充分大的初始温度$T$,合适的初始解$x_0$和迭代数$L$
    \item 对当前解施加扰动\[x_n' = C(x_n, T)\]
    \item 计算内能差\[\varDelta E = E(x_n') - E(x_n)\]
    \item 按以下概率接受新解$x_{n+1}=x_n'$,否则接受原解$x_n$ \[\mathbb{P} (\varDelta E )=\left\{ 
        \begin{array}{ll}
            1 & \text{, }\varDelta E < 0\\
            e^{-\frac{\varDelta E }{T}} & \text{, }\varDelta E \geq 0
        \end{array} \right.\]
    \item 判断当前解$x_{n+1}$是否符合收敛标准或达到迭代数上限。若其中任一满足,则进行下一步,反之则回到第二步。
    \item 判断当前温度是否达到温度下限。若达到,则结束程序;反之则对温度进行变换$T' = \text{TransT}(T)$后回到第二步。
\end{enumerate}

\subsection{注意事项}

\begin{itemize}
    \item 降温过程应当足够缓慢。虽然这会降低性能表现,但可以保证足够的准度。(当然,也不能慢过穷举)
    \item 初始温度应当足够大,否则可能一开始就只能困在局部最优解中无法出来。
    \item 显而易见,这个算法是求最小值的。因此欲求最大值时可以考虑将代价函数取相反数或取倒数(不变号的话)。
    \item 值得注意的是,扰动函数$C$没有一个固定的准则,其通常是因地制宜的,但这样的扰动一定是随机的。
    \item 本文所附代码(即SA文件夹中的)经实践证明该算法可以通过合理调整参数在短时间内大概率找到最优解邻域内的点。
\end{itemize}

\section{遗传算法(Genetic Algorithm)}

\subsection{想法}

它和SA一样是启发式算法。这个算法从进化论中抽象了几个对象与过程:染色体、基因、繁殖(组合交叉)、变异、自然选择。鉴于在自然选择中,
存活下来的总是相对适应环境的生物,我们将可行解抽象为生物,将待解的方程抽象为自然选择,在实践中,我们会对待解方程做一些变换,
使其成为非负的“适应度函数”。我们让生物发生交配,让染色体组合交叉从而产生新解;并在此过程中以一定几率发生变异产生新的基因,
从而模拟自然选择的过程。有时我们还会加入“精英策略”,即让适应度最高的几位被直接保留(考虑种马、最适者长寿等情况)。最终借助这个策略求出(近似)最优解。

\subsection{流程}

\subsubsection{全览}

\begin{enumerate}
    \item 首先生成一个初始种群,这个种群是随机的,一般来说服从均匀分布。
    \item 让这个种群开始“繁殖”,即先将种群内的成员生成后代时\textbf{按一定几率}进行组合交叉
        (这里之所以不是100\%发生是因为杂种优势不是必然的,从保证算法的有效性而言需要这样的保证。但后代的产生是必然的,
        只不过不交叉的话就保留本体的基因型),再让每个后代按较小的概率发生突变
    \item 根据适应度函数计算各成员的适应度,再根据适应度对种群个体进行选择。在这里一般应用最简单的轮盘赌选择,如需优化遗传算法,可以改进选择规则。
    \item 若满足“达到迭代次数”和“满足收敛准则”中的任何一个,那么退出流程,输出最优近似解;反之则回到第二步。
\end{enumerate}

\subsubsection{编码}

一般来说,为了方便交叉组合与突变的操作,我们不能直接对可行解操作,特别是针对自变量少的求函数极值的问题;对于旅行者问题,
单纯地对任意一对解进行交叉组合很容易产生可行域之外的解,也不符合要求,因此我们需要一定的编码方式来把当前的解转换为易于算法操作的码。
注意,这个编码目的是为了方便操作,这就要求它易于转换,位数多且对应的解码过程必须是单射(编码不一定是,因为算法主要操作对象是码,
不要求从可行解对应一个唯一的码的操作)。因此我们有两种常见编码模式:(还有一个符号编码更为抽象,同时也更加可以“望文生义”,故不介绍)

\paragraph{二进制编码}

如其名,二进制编码主要表现形式即为二进制数字串。我们记需要表示的数的区间为$[x_{min}, x_{max}]$,用\textbf{n}位二进制数表示(事实上位数是由最终欲求的解的精度决定的)
那么对于任意n位二进制数字$b_nb_{n-1}\ldots b_1$,我们可以解码为:
\begin{equation*}
    x = x_{min} + \sum\limits_{i=1}^nb_i2^{i-1}\times\frac{x_{max}-x_{min}}{2^n-1}
\end{equation*}
原因是显而易见的。

\paragraph{浮点数编码}

如其名,由实数列组成的编码。它事实上包含着许多种编码方案,只不过这些方案的唯一共性就是用了实数列。比如表示自变量数量较大的多元函数的可行解、
利用实数的排序关系表示一个有序集(即编码\{2,1,3\}为\{0.49, 0.12, 0.56\})等等,是一种比较灵活的编码,在此不做赘述。

\subsubsection{交叉与突变}

交叉所抽象的是繁衍,但事实上在代码中比起减数分裂,它的表现更像发生概率提高后的联会。常见交叉方法有:
\begin{enumerate}
    \item \textbf{单点交叉} 在一对染色体上随机地选择一个点,在该点后段交换两个配对个体的部分染色体
    (事实上出于限制问题规模的需求,子代数和父代数应当是一致的。故我们只需关心子代生成时是否经过交叉环节,而无需关系是否是”交换“产生的)
    \item \textbf{多点交叉} 即随机在染色体上取$n(n\geq 2)$个点,然后交换这$n$个点围成的染色体片段。
    \item \textbf{均匀交叉} 即染色体上的每个基因等概率地发生交换
    \item \textbf{算数交叉} 即子代是父系的线性组合,一般用于浮点数编码。
\end{enumerate}
而变异则是对自然中的变异过程的抽象,在表现形式上也与直觉相符。常见的突变方案有:
\begin{enumerate}
    \item \textbf{基本位变异} 染色体上的随机一个或多个基因发生变异(对于二进制编码来说即取反)
    \item \textbf{均匀变异} 如其名,染色体上的每个基因进行一次变异的判定。
\end{enumerate}
这些操作的核心目的都是为了对可行解进行扰动,并提高基因丰富性,避免陷入局部最优解中。

\subsubsection{适应度与自然选择}

\paragraph{适应度函数}
一般来说,适应度函数不会与目标函数一致,而是与目标函数保持正相关。因为适应度在数值上决定了当前个体在自然选择中的存活概率,
鉴于$\mathbb{P} $是一个测度,适应度应当是一个正数。所以可以考虑将目标函数值整体平移,即在一轮的有限个体内找到一个最小值,
使全体成员减去这一最小值并加上一个正实数,从而保证全体采样点均为正数。
记函数点值集为$f(X)=\left\{f(x_i):x_i \in X\right\} $,$X$为当前轮的所有染色体的集合,则按上述规则构建的适应度函数为:
\[f_{adjust}(x)=f(x)-\min f(X) + C,\, C > 0,\, x \in X\]

\paragraph{自然选择}
如其名,让种群内个体适者生存。在实际操作中,为了让种群内总数不变,“选择”操作的算法表现更类似于不适者被替换为适者的过程。这样的选择算法是较为多样的:
\begin{enumerate}
    \item \textbf{轮盘赌选择} 是一种有回放的随机采样方法。可以理解为从服从多项分布的整体中得到一个样本列。
    \item \textbf{随机竞争选择} 这种采样方法是一种对轮盘赌选择的优化:按以下规则选取每一个下一代的个体:每次按轮盘赌选择一对个体,然后在这一对中选取最高适应度的一个。
    \item \textbf{最佳保留选择} 这种采样方法也是一种对轮盘赌选择的优化:指定下一代的$k$个个体直接遗传上一代经过交叉、变异后适应度最优的前$k$个个体,其余按轮盘赌选择选取。
    \item \textbf{最佳保存选择} 指定下一代的$k$个个体不经变异、交叉直接遗传上一代的适应度最优的$k$个个体,其余按其他选择方式选取。
\end{enumerate}
还有许多选择策略,由于笔者暂时写不出其代码,暂时不列出。

\subsection{注意事项}



\end{document}