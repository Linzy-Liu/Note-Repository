\documentclass{ctexart}
\usepackage{amsmath, amsthm, amssymb, mathrsfs,anyfontsize}

\begin{document}

\section{模拟退火(Stimulated Annealing)}

\subsection{想法}

事实上,\textbf{模拟退火算法}可以视为一种改良后的``贪婪''类算法。它的重大进步就在于通过Metropolis采样让算法能脱离局部最优解,
得到全局最优解。这个算法抽象自金属冶炼的退火过程。被冶炼的物体在由高温缓慢降至低温时,物体自身会自发地使自身的内能降至最低点。
其原理在于高温时,物体内的原子会进行较强的分子热运动,易于跳出自身原来所在的平衡位置。而随着温度的缓慢降低,分子的跳跃会越来越困难。
从而最终达到``全局最低内能''的状态。

\subsection{流程}

\noindent 需给条件:\\
\begin{tabular}[t]{l|l}
    \hline
    初始解 &  $x_0$\\
    迭代数 & $L$ \\
    扰动函数 & $C$\\
    代价函数(内能函数) & $E$\\
    降温函数 & $\text{TransT}$\\
    \hline
\end{tabular}


\begin{enumerate}
    \item 给定充分大的初始温度$T$,合适的初始解$x_0$和迭代数$L$
    \item 对当前解施加扰动\[x_n' = C(x_n, T)\]
    \item 计算内能差\[\varDelta E = E(x_n') - E(x_n)\]
    \item 按以下概率接受新解$x_{n+1}=x_n'$,否则接受原解$x_n$ \[\mathbb{P} (\varDelta E )=\left\{ 
        \begin{array}{ll}
            1 & \text{, }\varDelta E < 0\\
            e^{-\frac{\varDelta E }{T}} & \text{, }\varDelta E \geq 0
        \end{array} \right.\]
    \item 判断当前解$x_{n+1}$是否符合收敛标准或达到迭代数上限。若其中任一满足,则进行下一步,反之则回到第二步。
    \item 判断当前温度是否达到温度下限。若达到,则结束程序;反之则对温度进行变换$T' = \text{TransT}(T)$后回到第二步。
\end{enumerate}

\subsection{注意事项}

\begin{itemize}
    \item 降温过程应当足够缓慢。虽然这会降低性能表现,但可以保证足够的准度。(当然,也不能慢过穷举)
    \item 初始温度应当足够大,否则可能一开始就只能困在局部最优解中无法出来。
    \item 显而易见,这个算法是求最小值的。
    \item 值得注意的是,扰动函数$C$没有一个固定的准则,其通常是因地制宜的,但这样的扰动一定是随机的。
\end{itemize}

\end{document}